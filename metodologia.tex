\externaldocument{desenvolvimento}
\chapter{METODOLOGIA}
\label{c.metodologia}

O estudo foi segmentado de acordo com as seguintes etapas:

\begin{itemize}
	\item[-] Levantamento bibliográfico
	\item[-] Análise e desenvolvimento das técnicas de detecção baseadas em assinatura utilizadas atualmente
	\item[-] Implementação Computacional
	\item[-] Análise dos resultados e conclusão da monografia.
\end{itemize}
A confecção da monografia ocorreu juntamente com o andamento das demais
etapas definidas durante toda a extensão do projeto. A ideia foi utilizar
máquinas virtuais e emuladores para implementar em caráter de teste os
algoritmos de detecção baseados em assinatura sobre um conjunto de programas
contendo alguns programas já infectados com amostras de \textit{malware} diversas para
observar, em termos gerais, a eficiência do tempo de execução dos algoritmos e
suas respectivas taxas de sucesso. Como o ambiente pode ser montado em qualquer
máquina que suporte a plataforma Windows e emulador Android, então não há
necessidade de que se reserve material da UNESP para uso exclusivo do projeto.
A figura 5 ilustra o planejamento do projeto:
\begin{figure}[H]
\caption{\small Estrutura do projeto}
\centering
\includegraphics[scale=0.4]{figs/fig2}
\label{f.estrutura_projeto}
\legend{\small Fonte: Elaborada pelo autor.}
\end{figure}

O fluxo de desenvolvimento do projeto consiste na aplicação de um índice de
regras de detecção do Yara, construído a partir da conversão de arquivos
contendo bases de assinaturas do Clam AV para arquivos contendo regras de
detecção do Yara, sobre o conjunto de amostras vivas de \textit{malware} obtidas
nos repositórios citados na seção do desenvolvimento do projeto. As varreduras
são feitas via linha de comando e para agilizá-las, além do uso de índices para
agrupar regras, também pode-se automatizar as varreduras em grupos maiores de
arquivos aplicando um pouco de \textit{shell script}. Os resultados das
varreduras serão armazenados em arquivos de texto que irão conter todos os
atributos analisados a cada arquivo a respeito de uma determinada regra. Vamos
explicar com um pouco mais de detalhe como ficam caracterizadas as infecções nos
\textit{logs} de resultados, com casos de teste envolvendo expectativas de
resultados positivos, falso-positivos e negativos. Ao final da seção de
desenvolvimento, também foi explicada a ideia da aplicação \textit{web} para
varreduras remotas e manipulação de regras do Yara.

\section{Materiais Utilizados}
\label{l.material}

\subsection{Servidor Local} % (fold)
\label{sub:servidor}

Foi utilizada uma máquina com 4GB de RAM, 1TB de armazenamento e sistema operacional Windows 10 de arquitetura 64 bits.

\subsection{Yara}
\label{sub:Yara}

Yara é uma ferramenta de código aberto recentemente desenvolvida, cujo intuito é
auxiliar desenvolvedores a identificar e classificar amostras de
\textit{malware}. Ele é construído em C, porém, com o auxílio de algumas
bibliotecas, há a possibilidade de incorporá-lo em scripts de outras linguagens,
como Python. Segundo ALVAREZ (\citeyear{yararules}), o funcionamento do Yara consiste em comparar
regras criadas pelo desenvolvedor que descrevem famílias de \textit{malware},
com o código de executáveis quaisquer, ou também de arquivos comprimidos em
alguns formatos comuns. As regras podem variar bastante em nível de complexidade
e mapearem diversas características do comportamento dos executáveis, que serão
detalhadas posteriormente. O projeto no momento encontra-se bastante popular e o
programa está sendo utilizado como ferramenta de construção de assinatura de
\textit{malware} por desenvolvedores de diversos grupos importantes no ramo,
entre eles:
\begin{itemize}
	\item[-] ActiveCanopy
	\item[-] Adlice
	\item[-] CrowdStrike FMS
	\item[-] Fox-IT
	\item[-] Heroku
	\item[-] Kaspersky Lab
	\item[-] Picus Security
	\item[-] ReversingLabs
	\item[-] Symantec
	\item[-] ThreatStream, Inc.
	\item[-] Trend Micro
	\item[-] VirusTotal Intelligence
	\item[-] We Watch Your Website
\end{itemize}

O Yara foi utilizado no presente trabalho de conclusão de curso como meio de
aplicação dos algoritmos de assinatura nos programas que serão examinados. Com
o auxílio de \textit{scripts} desenvolvidos em Python, será simulada uma varredura nesse
conjunto de programas aplicando regras obtidas através de sites como o
\href{yararules.org}{Yara Rules} e o
\href{http://sanesecurity.com/usage/signatures/}{Sane Security}, que
disponibilizam sem custo algum assinaturas de milhões de códigos maliciosos já
capturados pela web. Com esse grande arsenal de assinaturas de \textit{malware}, é
possível testar com certa completude as técnicas e assinaturas que são
desenvolvidas no mercado e utilizadas nos sistemas de detecção de intrusão.

\subsubsection{Regras de detecção do Yara}
\label{l.regrasdyara}
O Yara funciona interpretando arquivos que contém características que possivelmente
apontam para o padrão dos elementos da família do \textit{malware} que deve ser
isolado. As regras são escritas com uma sintaxe simples que remete à
construção de uma estrutura de dados em C, contendo strings a serem comparadas
tanto em formato texto quanto hexadecimal, conjuntos de strings, endereços da
memória que ficarão sob monitoramento, limitações de tamanho de arquivo,
expressões regulares, pontos de entrada de execução e variáveis externas.
Todos esses dados do arquivo examinado podem ativar condições, listadas na
construção dessas regras, que vão determinar se o arquivo corresponde ou não à
família definida nas regras de detecção. Para exemplificar mais claramente a
construção e o funcionamento das regras, seguem alguns exemplos de código
obtido do repositório do projeto \href{yararules.com}{Yara Rules} (yararules.com) no
\href{github.com}{Github}, ilustrados nas figuras 6 e 7.

\begin{figure}[H]
	\centering
	\caption{Regra de detecção para o \textit{malware} ``Zeus''.}
	\includegraphics[width=0.95\textwidth]{figs/zeus}
	\label{f.regrazeus}
	\legend{Fonte: elaborada pelo autor}
\end{figure}

Na imagem acima ilustra-se um exemplo de uma descrição para o \textit{malware} chamado
``Zeus'' (também conhecido como ``RPG'', ``ZBot'', ``Infostealer'', entre
outros), que é membro da família dos \textit{Trojans}. Segundo informações da
\href{www.wiki-security.com}{Wiki Security} \footnote{www.wiki-security.com} (\citeyear{wikisecurity}) os danos
causados por ele incluem roubo de informações pessoais como senhas de email e de
bancos, para posteriormente efetuarem-se transferências para ``contas-mula'' que
lavariam o dinheiro das vítimas até que ele chegasse aos atacantes. O
\textit{malware} foi escrito majoritariamente em C com alguns pequenos trechos
em php, e seu código fonte, vazado em 2011, pode ser encontrado neste outro
repositório público do Github (https://github.com/Visgean/Zeus). Como
pode-se observar, a regra escrita para marcar o trojan verifica os dois
primeiros \textit{bytes}, \textit{strings} e o protocolo utilizados por uma
determinada versão do ``Zeus'' e fica ativa quando a primeira e uma das outras
duas condições são satisfeitas.

Outro exemplo de regra de detecção envolvendo parâmetros semelhantes aos
anteriores é a que descreve o \textit{malware} ``Lenovo
Superfish'' (LENOVO, \citeyear{lenovosuperfish}) ; na verdade esse é o nome de um serviço
que vinha embutido em algumas séries de equipamentos lançados pela fabricante
Lenovo entre 2014 e 2015, cuja meta era ajudar consumidores a encontrar produtos
similares aos que eles buscavam pela web. Esse serviço possuía uma
vulnerabilidade de protocolos que foi explorada na criação do \textit{malware}
homônimo. No caso, as características que estão sendo mapeadas são novamente os
dois primeiros \textit{bytes} do programa, porém com strings em formato e
encodificações diferentes das encontradas na regra anterior, como segue na figura 7:

\begin{figure}[H]
	\centering
	\caption{Regra de detecção do ``Superfish''}
	\includegraphics[width=0.95\textwidth]{figs/superfish}
	\label{f.superfish}
	\legend{Fonte: elaborada pelo autor}
\end{figure}

As regras que serão aplicadas no decorrer do projeto serão unificadas num arquivo
de índice, compiladas com o uso de um \textit{script} em Python para a
interpretação do Yara e combinadas com o conjunto completo de arquivos de uma só
vez, pois o próprio Yara já possui métodos implementados que possibilitam
varreduras de maneira recursiva.

\subsection{\textit{Logs} de resultados do Yara} % (fold)
\label{sub:logs_de_resultados_do_yara}

As varreduras efetuadas pelo Yara podem ter seus resultados gravados em arquivos
de texto para análise posterior, que podem também denominarem-se
\textit{logs}, que por sua vez retornam um  breve dicionário de strings com formato exemplificado no Algoritmo 1:
\renewcommand{\thelstlisting}{\arabic{lstlisting}}
\begin{lstlisting}[caption=Conteúdo dos arquivos de resultado de varredura, label=resultyara]
{
'tags': ['mal', 'ware'],
'matches': True,
'namespace': 'default',
'rule': 'regras_compiladas',
'meta': {},
'strings': [(81L, '$a', 'abc'), (141L, '$b', 'def')]
}
\end{lstlisting}

Respectivamente, segue o que cada parâmetro representa:
\begin{itemize}
	\item '\textit{tags}': conjunto dos nomes dos programas maliciosos que estão sendo buscados

	\item '\textit{matches}': True significa positivo para infecção
	
	\item '\textit{namespace}': Nome do conjunto ao qual a regra escrita pertence (padrão: \textit{default})

	\item '\textit{rule}': Arquivo com as regras compiladas para aplicação

	\item '\textit{meta}': metadados do arquivo

	\item '\textit{strings}': conjunto de strings para comparação dentro do objeto de varredura
\end{itemize}

Existe também a possibilidade de utilizar-se outros módulos para verificação e
isolamento de outros atributos do arquivo como, por exemplo, detecção do uso de
compressão de executáveis, que é uma técnica comum empregada no desenvolvimento
de \textit{malware}. Por questões de viabilidade e complexidade do projeto,
procurou-se aplicar uma abordagem mais simples, sem a realização de testes de
implementação dos referidos módulos.

\subsection{Python}
\label{sub:python}

É uma linguagem de programação cujo desenvolvimento ocorre desde 1989. No ano de
1994, foi lançada a primeira distribuição da versão 1.0 oficial da linguagem e
sua versão mais recente foi lançada em setembro de 2015. Para as necessidades do
presente projeto, foi adotada a versão 2.7 da linguagem, por suportar a
implementação dos \textit{scripts} do Yara e também ser a distribuição mais
estável da linguagem, segundo o \textit{site} oficial da linguagem.\footnote{www.python.org} O
Yara suporta integração com scripts feitos tanto em código C puro como em
Python, porém a escolha da linguagem se deu principalmente pela quantidade de
material disponível, onde a grande maioria de scripts já prontos e documentados
foi feita em Python,  e pelo fácil acesso à comunidade de desenvolvimento caso
fosse necessário. As implementações em C são mais comuns em âmbito corporativo,
onde se constroem módulos mais otimizados para uso em varreduras de
\textit{malware} numa escala mais elevada e o desempenho e a confidencialidade
do código tornam-se um fator mais significativo na aplicação das regras.

\subsection{Clam AV}
\label{sub:clamav}

O Clam AV é um antivírus de código aberto que possui um sistema muito
interessante para a montagem de sua base de dados de assinaturas. O
\textit{Community Threat Tracking System} funciona coletando os dados de
infecções ocorridas nos sistemas dos usuários e aprimorando suas assinaturas
automaticamente com base nos dados estatísticos que a comunidade fornece. As
bases buscam evidenciar quais \textit{malwares} estão mais ativos no momento, e
tornam todas as suas descobertas públicas. Os usuários têm a opção de enviarem
dados tanto anonimamente, quanto com um identificador único para cada
instalação, e tais dados não são comercializados com terceiros, ou monetizados
de alguma outra forma, todas as análises feitas por eles são apenas um serviço
gratuito prestado à comunidade \textit{open source}. No projeto que foi
desenvolvido, utilizaram-se bases de dados com assinaturas utilizadas pelo Clam
AV em suas próprias varreduras em \textit{scripts} que as reconhecem,
descomprimem e convertem para regras do Yara, que a partir de então podem ser
editadas ou testadas isoladamente a critério do desenvolvedor.
