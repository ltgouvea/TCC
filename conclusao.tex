\chapter{CONCLUSÃO}
\label{c.conclusao}

A questão do aprimoramento das técnicas de segurança de redes e de informação
permanecerá relevante por muito tempo, em virtude do fato de que estamos
inseridos em um mundo que cada vez mais produz e depende de tecnologia e também
apresenta este mesmo comportamento quanto ao volume e à variedade de dados que
trafegam pela \textit{internet}. Com os estudos levantados e uma análise do
cenário atual da computação, pode-se afirmar que as técnicas estudadas e
aplicadas no desenvolvimento deste projeto estão, em termos de desempenho,
chegando ao seu máximo, pois a manipulação de assinaturas de \textit{malware}
implica características no processo de detecção que limitam uma possível
abordagem com o uso de técnicas de programação mais inovadoras. Embora a
eficácia das ferramentas atualmente empregadas seja satisfatória e as mesmas
apresentem uma boa precisão de detecção, estamos sempre atuando num contexto do
que pode-se chamar de `pós-infecção' ou pelo menos de uma infecção iminente,
onde os arquivos maliciosos já estão trafegando por pontos de rede ou até mesmo se encontram
latentes dentro dos ambientes vulneráveis infiltrados, seja por estarem à espera de um
comando do usuário ou por terem execução agendada. O panorama dos estudos nesse
segmento da área de segurança de redes está, em maior parte, tomando um viés
para a prevenção de infecções e episódios de vulnerabilidade, onde os métodos de
detecção de intrusão por anomalia e comportamento apresentam a vantagem de
estarem constantemente procurando se antecipar a possíveis ataques e a novas
formas de quebra de segurança, com a utilização de conceitos como redes neurais,
grafos, e até mesmo aprendizado de máquina.

No âmbito mais prático do desenvolvimento do projeto, observou-se que a abertura
para a criação e melhoria de técnicas envolvendo assinaturas é relativamente
recente, e as ferramentas encontradas e utilizadas para tais fins não são
voltadas para o público mais leigo e, talvez, curiosos e entusiastas do assunto,
haja visto que mesmo uma ferramenta como o Yara, cujo intuito é tornar essa
manipulação de assinaturas mais simplificada, requer uma configuração que exige
um razoável conhecimento de utilização de terminais e linhas de comando, e
afinidade com alguma linguagem de programação de alto ou médio nível para a
construção dos scripts de varredura ou conversão de bases de assinaturas. Foi
pensando nesse aspceto que se desenvolveu a ideia da aplicação \textit{online}
com um sistema capaz de automatizar as tarefas de manuseio de assinaturas de
\textit{malware}, para que as pessoas possam, além de testar a confiabilidade de
arquivos encontrados na \textit{internet}, ter um ponto de partida para conhecer
essa área da computação e também tentar colocar um pouco desse conhecimento em
prática de uma maneira colaborativa. O código-fonte do protótipo da aplicação
encontra-se hospedado \href{https://github.com/ltgouvea/TCC }{neste
repositório},  aberto para quaisquer desenvolvedores que queiram dar andamento
ao projeto caso se interessem. Já há serviços de varredura de \textit{malware}
gratuita disponíveis pela \textit{internet}, então o interesse maior no
desenvolvimento dessa aplicação é o diferencial que existiria na manipulação de
assinaturas de modo colaborativo, possivelmente anônimo e montando bases de
dados automaticamente no módulo de conversão de regras descrito no capítulo de
desenvolvimento do projeto. O resultado final foi uma aplicação funcional das
técnicas propostas no projeto que pode ser utilizada isoladamente, sem o auxílio
de outros \textit{softwares} antí-virus quaisquer, e posteriormente ampliada ou
incorporada no segmento de uma aplicação \textit{online} similar à ideia descrita
durante o desenvolvimento do projeto.
