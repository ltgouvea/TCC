\chapter{DESENVOLVIMENTO}
\label{c.Desenvolvimento}

O desenvolvimento do projeto dividiu-se em quatro fases: a montagem do ambiente de testes de assinaturas construídas
a partir de regras do Yara; a obtenção das regras e construção de um índice de agrupamento das mesmas para agilizar a
execução das varreduras; obtenção e escolha de amostras de \textit{malware} para testagem; bateria de testes final e
análise de resultados. Complementarmente, foi desenvolvida a ideia de uma aplicação \textit{web} para análise
de \textit{malware} com base em regras de detecção do Yara.

\section{Obtenção das regras de detecção e das amostras de \textit{malware}}
\label{s.obtregras}

A primeira etapa da implementação computacional do projeto foi a construção de um índice de regras do Yara compilado a partir das
bases de assinaturas do ClamAV e de outros conjuntos avulsos e menores de assinaturas disponibilizados em repositórios de código
aberto; em paralelo, também construiu-se um conjunto de algumas amostras vivas e selecionadas de \textit{malware} para a testagem
do índice sobre arquivos infectados. Foram selecionadas apenas amostras de \textit{malware} `potencialmente inofensivo', como
\textit{trojans}, que dependem de um comando de execução disparado por um usuário descuidado, para que não houvesse risco de
propagação e infecção do próprio ambiente local e de redes com as quais o ambiente local se comunicasse. Por isso, com fins de
prevenção, optou-se pela não realização de testes em, por exemplo, \textit{worms}, que conseguem se propagar de maneira independente
pelos ambientes onde chegam. Para converter as assinaturas, foi adaptado um \textit{script} em Python que interpreta os arquivos no
que contém as assinaturas do ClamAV, cujo formato é .cvd, e reescreve-as na sintaxe amigável de regras do Yara. No final da seção
de desenvolvimento, há uma lista com os \textit{links} para \textit{download} de todo o código-fonte do projeto. Após a tradução
das regras contidas nas bases de dados, é necessário compilá-las antes de finalmente ser possível aplicá-las numa varredura
real. Um outro pequeno \textit{script} em Python é capaz de realizar essa tarefa, vide código-fonte exibido na figura abaixo:
\begin{figure}
  \includegraphics[scale=0.6]{figs/script_conversao}
  \centering
  \caption{Script para compilação das regras de detecção}
  \label{f.script_comp}
\end{figure}


\section{Testes iniciais com o ambiente de desenvolvimento}
\label{s.testesiniciais}

Apenas com a finalidade de atestar o funcionamento da ferramenta estudada no projeto,
antes de decidir seguir com seu uso no decorrer da implementação planejada, algumas regras
de teste simples foram aplicadas sobre alguns arquivos de texto contendo pedaços de
código e um conjunto de \textit{strings} que deveria ser descoberto numa varredura. Os testes
foram bem sucedidos nesta etapa pois as regras escritas eram pequenas e rudimentares e também
ainda não havia necessidade de conversão de bancos de assinaturas ou montagem de
índices mais complexos e adaptação de \textit{scripts} envolvidos neste processo inicial.
Numa tentativa mais condizente com um cenário real de detecção de \textit{malware},
se fez a construção de um índice com um número bem maior de regras obtidas nos repositórios
já citados previamente. Contudo, nesta segunda etapa ainda não procurou-se converter as
regras contidas dentro das bases de assinaturas do ClamAV, por motivos de complexidade
e de completude dos testes propostos. Nessa mesma fase, as regras foram aplicadas dentro
de um conjunto mais numeroso de arquivos, já contendo alguns arquivos de código-fonte
e também executáveis de \textit{malware} misturados em algumas pastas, para ver como
o Yara funcionaria num contexto de varredura mais recursivo. Estas duas primeiras etapas
de testes foram bem sucedidas e o Yara acusou corretamente quais arquivos continham
as assinaturas apontadas pelo índice de regras aplicado.



\section{Testagem completa e análise de resultados}
\label{s.testefull}

\section{Ideia de uma aplicação web do projeto desenvolvido}
\label{s.prototipo}

Foi elaborada a ideia de um serviço capaz de receber regras do Yara ou arquivos de bases de assinaturas de \textit{malware}
para manter sua própria base de dados unificada e também capaz de realizar varreduras em arquivos carregados pelos usuários
ao servidor da aplicação. Na imagem a seguir temos um esboço do funcionamento da aplicação projetada.

\begin{figure}[H]
  \includegraphics[scale=0.6]{figs/flux_prototipo}
  \centering
  \caption{Fluxograma de funcionamento da aplicação. Fonte: elaborada pelo autor.}
  \label{f.flux_prototipo}
\end{figure}

Pensou-se também numa possível estrutura de comunicação entre os módulos utilizando
os recursos de uma API \textit{RESTful}, que cobriria as funções de transação com
o banco de dados, execução de scripts na camada do servidor e de \textit{upload}
e devolução de arquivos entre clientes e servidor, como por exemplo o recente projeto
Laravel PHP. Pode-se sugerir que A API troque dados com um \textit{front-end} implementado em cima
do \textit{framework} Angular JS, atualmente disponível em sua segunda versão,
pelas facilidades que ele contém para o desenvolvimento
de aplicações que possuem arquitetura semelhante à apresentada na ideia descrita.
Sugere-se o emprego de tais tecnologias nessa implementação pois são ótimas para se trabalhar com arquivos de texto,
por já incluirem bibliotecas para tratamento e interpretação de arquivos seguindo
diversos parâmetros para combinação e verificação de \textit{strings} sob padrões e
expressões regulares, que automatizariam a formatação e validação dos arquivos,
assegurando o funcionamento correto da montagem dos índices de assinaturas e dos
\textit{scripts} de varredura de arquivo executados na camada do servidor.
% TODO
% Listar estrutura de pastas do módulo de conversão e varredura
% Detalhar o funcionamento do desgraçado
% Mostrar o front-end e explicar como funcionaria o backend
% Escrever a conclusão
% Mandar para o Kelton Revisar
% Rezar para os deuses do TCC me olharem com bondade
% Comprar 5 fardos de cerveja
% 2 KG de carne
% 50g de aipim jamaicano
% Party hard.
