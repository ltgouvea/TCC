\chapter{OBJETIVOS}
\label{c.objetivos}

 

\subsection{Objetivo Geral}
\label{s.objetivog}

Analisar o resultado da ação de diferentes técnicas de detecção de \textit{malware}
baseadas em assinatura num ambiente de testes para obter métricas relevantes
quanto à eficiência nas varreduras realizadas, número de falsos positivos
encontrados e se possível determinar qual delas é a mais apropriada para uso
em ambientes mobile e de redes de computadores em geral.
 
\subsection{Objetivos Específicos}
\label{s.ojetivosp}

- Para embasar o projeto, foi levantado um histórico sobre segurança em redes
  de computadores e assuntos correlacionados principalmente às técnicas de
  detecção a serem estudadas.

- Montar um ambiente para que ele seja populado com conjuntos de programas a
  serem investigados e executar diferentes algoritmos para que posteriormente
  classifiquem-se seus resultados. Por questões de relevância, o escopo do
  trabalho envolverá sistemas \textit{Windows} e \textit{Android} pois a maioria dos usuários
  comuns utiliza estes sistemas operacionais.

- Comparar todos os métodos previamente escolhidos e testados para apresentar
  suas respectivas características e desempenhos.


\subsection{Justificativa}
\label{s.justificativa}

O desenvolvimento de novas técnicas de segurança computacional, bem como a
análise e aprimoramento de técnicas já construídas anteriormente, representam
sempre um avanço na área de segurança de redes, pois todo tipo de sistema
operacional possui alguma deficiência no quesito de segurança, haja visto que
mesmo agências de inteligência como a NSA enfrentam de tempos em tempos
problemas envolvendo vazamentos de dados e intrusões. De acordo com Hassan e
Muhammad (\citeyear{hassan10}), os tipos de ataque mais poderosos e bem
sucedidos ocorrem de dentro da rede, pois em diversas situações, o
administrador de rede confia em todos os usuários internos e não suspeita de
ataques vindos de seu próprio lado. Por isso, é sempre um investimento
desenvolver e manter uma política de aprimorar as defesas das redes de
computadores. O contexto do projeto é o da esfera do usuário comum, que
utiliza dispositivos comuns mas também é alvo de ataques que às vezes são
menores e até inofensivos, que contudo, ocorrem com frequência.

